\nopagenumbers
This document roughly follows the flow of the ``intro to tex'' document

Here is my first \TeX\ sentence.
I was the one who typeset it!

This is a new paragraph (line).
% This is a comment

This is a percentage sign: \%

This is a dollar sign: \$

This sentence\ \ \ has extra spaces in it - the control symbol used to
insert extract spaces is also good for
terminating control words

Single newlines
don't do anything except insert a space;

but double newlines get translated to one newline. How about triple?


Or quadruple?



Man shoo
\par You can also make a new paragraph like this

Here are some special characters: 

$\backslash$

$\{$ (open group)

$\}$ (close group)

\% (comments)

\& (tabs, table alignments)

\~{} (tilde, unbreakable space)

\$ (start or end math text)

\^{} (math superscript)

\_{} (math subscript)

\# (define replacement symbols)

Here's how to write accented words (the control character precedes the
letter to be accented): premi\`ere (notice the grave accent -
on my keyboard it's the same key as tilde). An acute accent is the regular
quote: fianc\'ee; cedille is typed with a space following the control word: 
c\c a va; umlaut uses double
quote, as in the New Yorker's spelling of co\"ordinate; tilde is easy,
jalape\~no. There are two more accents that don't exist in any language I know; this
is called a macron: \=o and uncreatively this is called a dot: \.o (using
the letter `o' for reference; the accent may be applied to any letter).
Here's a dotless \i (letter `i' - looks strange, don't it), which is good 
for putting accents on: \'\i. Dotless lowercase j also exists

In addition to the cedille, you have to put a space after the control word
for the following (using the letter `o' again as an example):
underdot: \d o; underbar: \b o; something called a h\'a\v cek: \v o;
something that looks similar but is called a breve: \u o;
something called a tie: \t oa (notice the double letters); and the Hungarian
umlaut: \H o

Some non-English letters/symbols:
\AE, \ae, \OE, \oe, \AA, \aa, \O, \o, \L, \l, \ss, ?`, !`, {\it\$}

Here are some dash-like characters: hyphen mother-in-law , en-dash 1--3, minus sign $-$3. The em-dash is used in the following sentence:
I ate all the pizza --- and fell right asleep.

Here's how quotes are supposed to work ``Wow!'' (ie, use the tilde and the
single quote) - single quote marks: `wow'

You can do ellipses like this: ... or like this: \dots (the latter is
preferred)

Spacing after dots is tricky: Mr. Pepperoni, Mrs.\ Tomato, Sr.~Sanch\'ez
\par
This is roman type (default), \it This is italics. \rm Now I've switched back.
\bf This is bold, \sl this is slanted, \tt this is typewriter, \rm and \cal
this is how math is formatted. \it Sometimes it is necessary to use a thing
called the italic correction, which is written \/ like that

\rm Here's how to enbiggen your text:
% define a new font called rmonehalf. cmr10 is the "official" name for the
% default typeface
\font\rmonehalf = cmr10 scaled \magstephalf
\font\rmone = cmr10 scaled \magstep 1
\font\rmtwo = cmr10 scaled \magstep 2
\font\rmthree = cmr10 scaled \magstep 3
\font\rmfour = cmr10 scaled \magstep 4
\font\rmfive = cmr10 scaled \magstep 5
\font\sf = cmss10  % san serif
\rmonehalf Sample

\rmone Sample

\rmtwo Sample

\rmthree Sample

\rmfour Sample

\rmfive Sample 
\par
\eject \vfill %means "new page"
\rm And here's a San Serif font:
\sf I love San Serif

\rm See page 18 of the ``intro to tex'' document for more fonts.
 
Skipping section 3 of the tutorial because it's boring, except that \hfil \break
can be used to force a newline

\hfil \break 
Does it work for two newlines? YES!

But be careful with hfils and breaks - they can cause underfull boxes. Best
practice is to surround them with a newline on each side.

\hfil \break
Section 4 is about groups, which are neat for
applying {\bf styles} to only the {\it grouped text}, but otherwise I'm gonna skip it
here

\hfil \break 
{\rmthree Math - the good stuff}

There are two different ways to show math: inline and display. An in-line
equation is $x=y+1$ and a display equation is $$x=y+1$$
Here's some things: $$C(n,r) = n!/(r!(n-r)!)$$ 
%$\alpha\beta= \gamma + \delta$$
$$52 + 6 \over c + \pi$$
Partial derivatives (note the grouping):
$${\partial \over \partial x} f(x,y)$$ 
Super and subscripting are done like so:
$$e^x, e^{-x}, e^{i\pi + 1} = 0, x_0, x_0^2$$
Sums, integrals, limits:
$${\sum_{k=1}^n k^2}, {\int_0^x f(t) dt}$$
$$\lim_{n\to \infty}({n+1 \over n})^2=e$$
Square root:
$$\sqrt{x^2 + y^2}$$
Nth root:
$$\root n \of {1+x^n}$$
Special functions: $\sin(2\theta) = 2\sin\theta\cos\theta$
Matrices: $$\pmatrix{
a&b&c\cr
e&f&g\cr
}$$
Piecewise-linear function:
$$ |x| = \left \{ 
\matrix{
x, & x \geq 0 \cr
-x, & x \leq 0 \cr
}
\right.  $$ % . after \right means "delete the right delimiter"
Vectors:

$\vec x \cdot \vec y = 0 $ if and only if $\vec x \perp \vec y$

Proofs and stuff:
$$\forall x\in \Re)(\exists y\in\Re$$such that $$ y>x$$
\bye
